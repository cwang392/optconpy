\documentclass[a4paper,10pt]{article}
\usepackage[T1]{fontenc}
\usepackage[utf8]{inputenc}
\usepackage{epsfig}
\usepackage{amssymb,amsmath,amsthm,mathrsfs,mathtools,amscd}
\usepackage{natbib}
\usepackage{float}
\usepackage[figtopcap]{subfigure}
\usepackage[bookmarks=true]{hyperref}



%\input{mathComAbb} %uses mathcomabb.tex to define commands for symbols and notations 

%opening
\title{}
\author{Jan Heiland}

\begin{document}
\maketitle
\tableofcontents
%\input{mathEnv}

\section{How to assemble the output}

How to compute $y(s) = \frac{1}{C}\int_{xmin}^{xmax} v(x,s) dx$?

I don't think there is a generic way to do this, since the (discrete) representation of $u(y)$ has to be defined beforehand. If you want it expressed in a 1D finite element basis then you can do the following

\begin{enumerate}
	\item define the 1D FEM basis $\{\phi_i(y)\}$
	\item for every $\{\phi_i(y)\}$ define a 2D expression $b_i(x,y) = \phi_i(y)$
	\item test every $b_i$ against your $f$ to get
		$$ \int \int b_i(x,y)f(x,y)dxdy=\int\phi_i(y)u(y)dy =: \tilde u_i
		$$
	\item compute the coefficients of the expansion of $u$ in the FEM basis - $u(y) \approx \sum_i u_i \phi_i(y)$ - via
		$$ \begin{bmatrix} \vdots \\ u_i \\ \vdots \end{bmatrix} = \mathcal M_{\phi}^{-1} \begin{bmatrix} \vdots \\ \tilde u_i \\ \vdots \end{bmatrix},
			$$
			where $M_\phi$ is the FEM mass matrix.
	\end{enumerate}

If you have the expression 'bi' in fenics and function 'f', then you find

    'ui' = assemble('bi'*'f'*dx)

\bibliographystyle{plain}
\bibliography{}
\end{document}

